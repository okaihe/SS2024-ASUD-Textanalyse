\documentclass[12pt]{article}

% Packages
\usepackage[a4paper,left=2.5cm,right=2.5cm,top=2.5cm,bottom=2.5cm,footskip=1cm]{geometry}
\usepackage{xcolor}
\usepackage[explicit]{titlesec}
\usepackage[colorlinks=false]{hyperref}
\usepackage[T1]{fontenc}
\usepackage{graphicx}
\usepackage{caption}
\usepackage{listings}
\usepackage{color}

% Colors
\definecolor{dkgreen}{rgb}{0,0.6,0}
\definecolor{gray}{rgb}{0.5,0.5,0.5}
\definecolor{mauve}{rgb}{0.58,0,0.82}
\definecolor{lightgray}{rgb}{.9,.9,.9}
\definecolor{darkgray}{rgb}{.4,.4,.4}
\definecolor{purple}{rgb}{0.65, 0.12, 0.82}

% Javascript

\lstdefinelanguage{JavaScript}{
    keywords={typeof, new, let, for, const, continue, of, await, true, false, catch, function, return, null, catch, switch, var, if, in, while, do, else, case, break, try, catch, throw, async },
    keywordstyle=\color{blue}\bfseries,
    ndkeywords={class, export, boolean, default, implements, import, this},
    ndkeywordstyle=\color{darkgray}\bfseries,
    identifierstyle=\color{black},
    sensitive=false,
    comment=[l]{//},
    morecomment=[s]{/*}{*/},
    commentstyle=\color{purple}\ttfamily,
    stringstyle=\color{dkgreen}\ttfamily,
    frame=none,
    morestring=[b]',
    morestring=[b]"
}

% Code
\lstset{frame=tb,
    aboveskip=3mm,
    belowskip=3mm,
    showstringspaces=false,
    columns=flexible,
    basicstyle={\small\ttfamily},
    numbers=left,
    keywordstyle=\color{blue},
    commentstyle=\color{dkgreen},
    stringstyle=\color{mauve},
    breaklines=true,
    breakatwhitespace=true,
    tabsize=4
}

% Settings
\hypersetup{pdfborder=0 0 0}
\definecolor{FomBlue}{HTML}{00998A}
\linespread{1.20}

% Section heading styling
\titleformat{\subsection} {\bfseries} {\thesubsection} {1em} {#1} [{\titlerule[0.2pt]}]
\titlespacing{\subsection} {0em} {3.3em} {0.8em}

% Title page
\title{%
    \textbf{Zusätzliche Beteiligung}\\
    \bigskip
    \large Analyse der Artikel des Nachrichtensenders NTV
}
\author{Kai Herbst, Manuel Zeh, Henrik Popp}
\date{August 2024}

% Document
\begin{document}
\begin{sloppypar}
	\maketitle
	\thispagestyle{empty}

	\newpage
	\setcounter{page}{1}
	\pagenumbering{Roman}

	\renewcommand{\contentsname}{Inhaltsverzeichnis}
	\tableofcontents

	\newpage
	\setcounter{page}{1}
	\pagenumbering{arabic}

	\section{Daten-Pipeline}


	% \vspace{0.5cm}
	% \begin{figure}[h]
	% 	\centering
	% 	\includegraphics[width=12cm]{images/TF-Player.jpeg}
	% 	\caption{Tf-Video-Player}
	% \end{figure}

	\appendix
	\section{Lambda-Funktion zur Sammlung der Artikel}

	\subsection{Index.mjs}

	Die Hauptfunktion, die von der Lambda-Funktion aufgerufen wird.

	\lstinputlisting[language=JavaScript,caption=index.mjs]{../ntv-lambda/index.mjs}

	\newpage
	\subsection{Categories.mjs}

	Die verschiedenen NTV-Kategorien, die abgerufen werden sollen.

	\lstinputlisting[language=JavaScript,caption=categories.mjs]{../ntv-lambda/categories.mjs}

	\newpage
	\subsection{S3Helper.mjs}

	Funktionen zum Zugriff auf die entsprechenden AWS S3-Buckets.

	\lstinputlisting[language=JavaScript,caption=s3Helper.mjs]{../ntv-lambda/s3Helper.mjs}

	\newpage
	\subsection{Helper.mjs}

	Hilfsfunktionen zum Verarbeiten der von NTV empfangenen Daten.

	\lstinputlisting[language=JavaScript,caption=helper.mjs]{../ntv-lambda/helper.mjs}



\end{sloppypar}
\end{document}
